\documentclass{cours}

\title{\textsc{Etude Computationnelle de la Stabilité Interlangue des Catégories Morphosyntaxiques}\\
{\small Rapport de Stage de L3} }
\author{Matthieu Boyer}

\newcommand{\codedir}{Morphosyntactic-Categories_Code}
\usepackage{nicematrix}

\begin{document}

    \begin{abstract}
        Dans ce rapport, nous nous intéressons à la stabilité interlangue des catégories morphosyntaxiques.
        Nous avons quantifié la manière dont différentes catégories descriptives d'un langage ont différentes
        significations dans différents langages,
        et particulièrement la manière dont un concept est matérialisé dans différents langages.
    \end{abstract}


    \section{Why ?}\label{sec:why-?}
    Martin Haspelmath sur la différence entre une catégorie linguistique descriptive dans un langage et une catégorie
    linguistique comparative dans le méta-langage:
    \begin{quote}
        There is a fundamental distinction between language-particular categories of languages (which descriptive
        linguists must describe by descriptive categories of their descriptions) and comparative concepts (which
        comparative linguists may use to compare languages).
    \end{quote}
    {\flushright
    {\textit{Martin Haspelmath}, \textsc{How comparative concepts and descriptive linguistic categories are different}}}
    Dans ce rapport, nous allons donc nous intéresser à la notion fondamentale de catégorie morphosyntaxique, et
    comparer les descriptions dans différents langages de catégories linguistiques comparatives.
    Pour ce faire, nous allons considérer que les relations de dépendances (\textit{reldep}
    ) décrites par les annotations de \textsc{Universal Dependencies} (UD)
    sont une manière de représenter des catégories comparatives.


    \section{Première Approche.}\label{sec:premiere-approche.}
    Nous considérons tout d'abord que chaque \textit{reldep} décrit une unique catégorie comparative et que plusieurs
    \textit{reldep} ne peuvent instancier une même catégorie comparative.
    En comptant le nombre d'instances de chaque \textit{reldep}
    pour un mot vérifiant une propriété grammaticale de la langue (i.e.\
    une catégorie descriptive, que l'on représente par une \textit{feature}
    d'UD, typiquement les cas pour des langues en utilisant), on obtient une représentation vectorielle des catégories
    descriptives et on peut donc mesurer la proximité de deux catégories descriptives dans deux langues différentes.
    Les corpus utilisés dans cette première partie sont ceux du projet \textsc{Universal Dependencies}, accessibles en ligne.

    \subsection{Avec la distance Cosinus}
    On trouve par exemple les résultats suivants:

    \renewcommand{\arraystretch}{1.1}
\begin{table}[H]
	\centering
	\resizebox{\textwidth}{!}{\begin{NiceTabular}{ccccccccc}
		Proximity with: & Abl & Acc & Dat & Gen & Ins & Loc & Nom & Voc \\
		First Quartile & 0.028 & 0.394 & 0.030 & 0.020 & 0.042 & 0.014 & 0.038 & 0.000 \\
		Median & 0.202 & 0.711 & 0.181 & 0.123 & 0.236 & 0.176 & 0.137 & 0.007 \\
		Third Quartile & 0.393 & 0.860 & 0.379 & 0.302 & 0.408 & 0.381 & 0.272 & 0.040 \\
		Mean & 0.242 & 0.616 & 0.236 & 0.196 & 0.255 & 0.230 & 0.188 & 0.042 \\
	\CodeAfter
		\begin{tikzpicture}
			\foreach \i in {1,...,6}
				{\draw[draw=vulm] (1|-\i) -- (10|-\i);}
			\draw[draw=vulm] (2|-1)--(2|-6);\end{tikzpicture}
	\end{NiceTabular}}
	\caption{Proximities for Case=Acc}
\end{table}

    \renewcommand{\arraystretch}{1.1}
\begin{table}[H]
	\centering
	\begin{NiceTabular}{cccccccc}
		Proximity with: & Case=Nom & Case=Acc & Case=Dat & Case=Gen & Case=Voc & Case=Loc & Case=Abl \\
		Median & 0.0 & 0.0 & 0.0 & 0.0 & 0.0 & 0.0 & 0.0 \\
		Mean & 0.03328 & 0.05376 & 0.09264 & 0.05258 & 0.00444 & 0.05714 & 0.0343 \\
		NLow & 57901 & 38008 & 28140 & 34565 & 21502 & 16842 & 8188 \\
		NHigh & 412 & 2446 & 9658 & 2324 & 12 & 7240 & 4837 \\
		First Quartile & 0.0 & 0.0 & 0.0 & 0.0 & 0.0 & 0.0 & 0.0 \\
		Third Quartile & 0.0 & 0.0 & 0.0 & 0.0 & 0.0 & 0.0 & 0.0 \\
	\CodeAfter
		\begin{tikzpicture}
			\foreach \i in {1,...,8}
				{\draw[draw=vulm] (1|-\i) -- (9|-\i);}
			\draw[draw=vulm] (2|-1)--(2|-8);\end{tikzpicture}
	\end{NiceTabular}
	\caption{Proximities for Case=Dat}
\end{table}

    \renewcommand{\arraystretch}{1.1}
\begin{table}[H]
	\centering
	\resizebox{\textwidth}{!}{\begin{NiceTabular}{ccccccccc}
		Proximity with: & Abl & Acc & Dat & Gen & Ins & Loc & Nom & Voc \\
		First Quartile & 0.037 & 0.020 & 0.022 & 0.032 & 0.056 & 0.027 & 0.026 & 0.000 \\
		Median & 0.198 & 0.123 & 0.134 & 0.317 & 0.249 & 0.188 & 0.104 & 0.006 \\
		Third Quartile & 0.416 & 0.302 & 0.341 & 0.823 & 0.449 & 0.400 & 0.225 & 0.047 \\
		Mean & 0.259 & 0.196 & 0.214 & 0.421 & 0.282 & 0.243 & 0.159 & 0.058 \\
	\CodeAfter
		\begin{tikzpicture}
			\foreach \i in {1,...,6}
				{\draw[draw=vulm] (1|-\i) -- (10|-\i);}
			\draw[draw=vulm] (2|-1)--(2|-6);\end{tikzpicture}
	\end{NiceTabular}}
	\caption{Proximities for Case=Gen}
\end{table}

    \begin{table}[H]
	\centering
	\begin{NiceTabular}{ccccccc}
		Proximity with: & Case=Nom & Case=Acc & Case=Dat & Case=Gen & Case=Voc & Case=Loc \\
		Median & 0.44588 & 0.46849 & 0.48794 & 0.46892 & 0.51927 & 0.57887 \\
		Mean & 0.49841 & 0.51443 & 0.51995 & 0.50402 & 0.54114 & 0.58953 \\
		NLow & 27886 & 21110 & 21518 & 21266 & 11000 & 18048 \\
		NHigh & 29971 & 27910 & 28499 & 25073 & 16744 & 45864 \\
		First Quartile & 0.23495 & 0.26545 & 0.27671 & 0.26711 & 0.26242 & 0.3383 \\
		Third Quartile & 0.77133 & 0.76979 & 0.7566 & 0.72731 & 0.83677 & 0.87388 \\
	\CodeAfter
		\begin{tikzpicture}
			\foreach \i in {1,...,8}
				{\draw[draw=vulm] (1|-\i) -- (8|-\i);}
			\draw[draw=vulm] (2|-1)--(2|-8);\end{tikzpicture}
	\end{NiceTabular}
	\caption{Proximities for Case=Loc}
\end{table}

    \renewcommand{\arraystretch}{1.1}
\begin{table}[H]
	\centering
	\begin{NiceTabular}{cccccccc}
		Proximity with: & Case=Nom & Case=Acc & Case=Dat & Case=Gen & Case=Voc & Case=Loc & Case=Abl \\
		Median & 0.0 & 0.0 & 0.0 & 0.0 & 0.0 & 0.0 & 0.0 \\
		Mean & 0.26295 & 0.07799 & 0.04112 & 0.06237 & 0.01181 & 0.02566 & 0.0162 \\
		NLow & 18404 & 80944 & 75574 & 79415 & 35362 & 48157 & 28185 \\
		NHigh & 54192 & 2105 & 506 & 2145 & 254 & 211 & 107 \\
		First Quartile & 0.0 & 0.0 & 0.0 & 0.0 & 0.0 & 0.0 & 0.0 \\
		Third Quartile & 0.54978 & 0.0799 & 0.01559 & 0.05144 & 0.0 & 0.0 & 0.0 \\
	\CodeAfter
		\begin{tikzpicture}
			\foreach \i in {1,...,8}
				{\draw[draw=vulm] (1|-\i) -- (9|-\i);}
			\draw[draw=vulm] (2|-1)--(2|-8);\end{tikzpicture}
	\end{NiceTabular}
	\caption{Proximities for Case=Nom}
\end{table}

    \begin{table}[H]
	\centering
	\begin{NiceTabular}{ccccccc}
		Proximity with: & Case=Nom & Case=Acc & Case=Dat & Case=Gen & Case=Voc & Case=Loc \\
		Median & 0.53652 & 0.52529 & 0.49388 & 0.55171 & 0.88359 & 0.56814 \\
		Mean & 0.55947 & 0.54856 & 0.53176 & 0.55858 & 0.76794 & 0.57436 \\
		NLow & 10168 & 9358 & 10354 & 8638 & 2688 & 6416 \\
		NHigh & 18219 & 14842 & 13198 & 14931 & 33512 & 12001 \\
		First Quartile & 0.27545 & 0.27187 & 0.24879 & 0.28231 & 0.63158 & 0.27575 \\
		Third Quartile & 0.87261 & 0.83938 & 0.81784 & 0.84647 & 0.97788 & 0.8884 \\
	\CodeAfter
		\begin{tikzpicture}
			\foreach \i in {1,...,8}
				{\draw[draw=vulm] (1|-\i) -- (8|-\i);}
			\draw[draw=vulm] (2|-1)--(2|-8);\end{tikzpicture}
	\end{NiceTabular}
	\caption{Proximities for Case=Voc}
\end{table}

    \renewcommand{\arraystretch}{1.1}
\begin{table}[H]
	\centering
	\begin{NiceTabular}{cccccccc}
		Proximity with: & Case=Nom & Case=Acc & Case=Dat & Case=Gen & Case=Voc & Case=Loc & Case=Abl \\
		Median & 0.0 & 0.0 & 0.0 & 0.0 & 0.0 & 0.0 & 0.0 \\
		Mean & 0.00599 & 0.00569 & 0.0167 & 0.0087 & 0.00033 & 0.01607 & 0.02558 \\
		NLow & 7840 & 6501 & 3013 & 6616 & 1545 & 2368 & 2056 \\
		NHigh & 49 & 453 & 2916 & 457 & 0 & 3704 & 6022 \\
		First Quartile & 0.0 & 0.0 & 0.0 & 0.0 & 0.0 & 0.0 & 0.0 \\
		Third Quartile & 0.0 & 0.0 & 0.0 & 0.0 & 0.0 & 0.0 & 0.0 \\
	\CodeAfter
		\begin{tikzpicture}
			\foreach \i in {1,...,8}
				{\draw[draw=vulm] (1|-\i) -- (9|-\i);}
			\draw[draw=vulm] (2|-1)--(2|-8);\end{tikzpicture}
	\end{NiceTabular}
	\caption{Proximities for Case=Abl}
\end{table}

    \subsection{Avec l'algorithme de \textsc{Zassenhaus}}
    On considère les espaces vectoriels engendrés par la représentation vectorielle du système de cas d'une langue, que l'on appellera \textit{espaces de cas}.
    Ceux-ci sont d'une certaine dimension finie.
    On applique alors sur toute paire de système de cas l'algorithme de Zassenhaus, permettant de générer une base de l'espace somme et de l'espace intersection.
    Toutefois, la grande variance au niveau des coordonnées, et la trop faible dimension (au plus 9) dans un grand espace (dimension 218), rend l'intersection toujours nulle numériquement.

    \subsection{Angle entre Cas et Système de Cas}
    On considère à nouveau la distance cosinus, mais cette fois-ci, non pas entre deux vecteurs, mais entre un vecteur et un espace de cas.
    Ceci est fait en considérant le projeté orthogonal d'un vecteur sur un espace de cas et en mesurant l'angle entre les deux (ou la distance cosinus).
    Ici intervient le premier résultat tangible de ce rapport: l'angle entre le vocatif du farsi et le système de cas arabe est de l'ordre de $10^{-16}$.
    En regardant de plus près les corpus farsis\footnote{et non pas les dindes.}, on observe que cela découle d'une idiosyncracie dans les annotations.
    En farsi, le lemme (unité morphologique abstraite: \textsl{fais} et \textsl{fait} sont deux graphies du même lemme \textsl{faire}, conjugué à deux personnes différentes)
    % \textit{ای}\footnote{Translit=āī}
    est décrit comme une interjection portant le vocatif et se reliant à un nom au cas absolu par la relation de transmission de cas (c'est à dire de marquer le cas pour un autre mot).
    Ce lemme agit donc en réalité plus comme une apposition.
    Le vocatif n'apparaît que très peu en farsi, et majoritairement dans cette situation.
    Ainsi, il semble que nous ne pouvons pas tirer d'enseignements du farsi vers une autre langue, et que par ailleurs, les corpus UD ne sont pas faits pour ce genre de traitement.

    \subsection{Distance Euclidienne}
    On considère cette fois la distance euclidienne entre tous deux vecteurs.
    Exemple Czech - Russe:
    \begin{tabular}{>{\tt}rcccc}
        &Dat RU & Gen CZ & Gen RU & Dat CZ\\
        Total & 1711 & 2631 & 2070 & 277\\
        obl & 450 & 208 & 219 & 48\\
        iobj & 340 & 0 & 0 & 0\\
        amod & 243 & 736 & 475 & 54\\
        nmod & 300 & 1000 & 980 & 24\\
        conj & 112 & 225 & 84 & 21\\
        case & 0 & 340 & 1 & 87\\
        det & 34 & 80 & 79 & 3
    \end{tabular}
    On enlève ensuite \texttt{amod, conj, det} puisque ces reldep démontrent l'accord vers la tête, et donc des doublons dans les données.
    On enlève aussi case, qui souvent (notamment visible dans l'exemple ci-dessous), est utilisé pour marquer le cas avec une apposition (sur, sous), et ceci dépend très fortement de la personne qui a annoté le corpus.
\end{document}
