\documentclass{cours}

\title{Rapport de Stage de L3\\
{\small \sc Etude Computationnelle de la Stabilité Interlangue des Catégories Morpho-Syntaxiques}}
\author{Matthieu Boyer}

\begin{document}

\begin{abstract}
	Dans ce rapport nous nous intéressons à la stabilité interlangue des catégories morpho-syntaxiques.
	Nous avons quantifié la manière dont différentes catégories descriptives d'un langage ont différentes significations dans différents langages,
	et particulièrement la manière dont un concept est matérialisé dans différents langages.
\end{abstract}<++>

\section{Why ?}
Martin Haspelmath sur la différence entre une catégorie linguistique descriptive dans un langage et une catégorie linguistique comparative dans le méta-langage:
\begin{quote}
	There is a fundamental distinction between language-particular categories of languages (which descriptive linguists must describe by descriptive categories of their descriptions) and comparative concepts (which comparative linguists may use to compare languages).
\end{quote}
\flushright{\textit{Martin Haspelmath}, \textsc{How comparative concepts and descriptive linguistic categories are different}}

\end{document}
