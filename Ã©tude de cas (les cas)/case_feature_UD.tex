\section{The \texttt{Case} Feature across Treebanks}
While realising this study, we stumbled upon a number of incongruities in the way the different corpus use the \texttt{Case} feature.

There are essentially three ways the feature \texttt{Case} is used in the UD treebanks.
The first and by far the most common use is to annotate inflected forms of nouns, pronouns and proper nouns in languages where these words inflect according to their role in a clause, as well as determiners, adjectives and participles in languages where they inflect to match the case of their governor.

The second use that is documented in UD's guidelines\footnote{See the page of the \texttt{Case} feature: \url{https://universaldependencies.org/u/feat/Case.html}.}, is to annotate adpositions with the case they give to their nominal phrase, especially so in languages without over case marking on nouns.
This annotation principle indicates that UD leans more toward the application of comparative concepts to individual languages.
Indeed, if a language does not use the case category, then the ``case'' represented by an adposition can only be inferred either by comparing its distribution to the distribution of actual cases in languages that possess that category, or by applying formal comparative definitions.

However, this is not always how this feature is used, as in Czech CLTT treebank \cite{cltt} for example, adpositions are annotated with the \texttt{Case} feature and their value always match that of their governing noun.
This is all the more surprising that Czech adpositions are invariable and can license several case values.

This indeed points to another problem with case annotation on adpositions.
Like languages exhibiting case syncretism\footnote{A given word form can be ambiguous as to its morphological features. For example, the Latin form \textit{rosae} can be either a genitive or dative singular or a nominative or vocative plural.}, adpositions can in principle also be used to mark different syntactic and semantic roles.
It becomes then even less clear how one should proceed in assigning cases to adposition.

The third and most divergent use of the \texttt{Case} feature can be seen in the Persian Seraji treebank.
In this treebank, we only find three case values : \texttt{Case=Loc}, \texttt{Case=Tem} and \texttt{Case=Voc}.
The first two values are exclusively used to annotate adverbs of place and adverbs of time respectively.
The third value is used to annotate an interjection used to create vocative noun phrase.

%Following on this method, we defined the vector spaces generated by the family of the vectors representing all the cases in a corpus (called \emph{case spaces}) and computing the cosine distance between two vectors, but this didn't yield any usable results. 
%Indeed the results were too blended in the number of results. 
%Moreover the density of data combined with the numerical precision needed for comparisons makes the results difficult to exploit manually. 
%We thus tried the following: computing the cosine distance between a vector and its projection on a case space. 
%This got us an aberration when looking at the projection of farsi vocative on arabic. 
%Indeed, farsi vocative was near orthogonal to arabic, while the two languages are close. 
%This actually comes from the fact that vocative in farsi is used to annotate some interjections used as adpositions, with the relation \texttt{case}. 