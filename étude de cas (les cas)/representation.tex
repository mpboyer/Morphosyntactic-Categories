\section{Case Representation}

In order to compare cases from different languages, we need to find a shared representation that should be as language agnostic as possible.
We decided to use the syntactic profile of a case defined as the probability distribution over the dependency relations to its governors.
This choice is both theoretical since core cases are usually defined in terms of syntactic relations to the other constituents of a sentence, and practical since UD treebanks are annotated with dependency labels.

In order to make the representations even more language agnostic, we decided to ignore relation sub-types since they are not consistently used across languages and corpora.
So, both \texttt{flat:foreign} and \texttt{flat:name} are counted as \texttt{flat}.

We give two representations to each case in a language.
The first is the empirical probability distribution of the relation of a word displaying that case to its governor.

However, there are several mechanisms underlying case assignment, and not all are as informative.
%The direct object of a verb will usually marked in the accusative case.
For example, when determiners inflect for case, they usually inherit their value from their head noun, which therefore does not teach us much about that case since a determiner can in principle take any case that way.
Similarly, it would artificially separate cases from languages with articles (a high proportion of \texttt{det} relations) from those of languages without.

Furthermore, as mentioned in the previous section, UD also allows annotation of the \texttt{Case} feature on adpositions, which is quite different from the way cases are generally assigned to nouns.
For all these reasons, we thus decided to have a part-of-speech based representation too.

The second representation is thus the syntactic profile of the nouns (\texttt{NOUN}) which bear the said case.
This gets rid of less informative dependency relations such as \texttt{case}, \texttt{amod} or \texttt{det} and we further decided to ignore the \texttt{conj} relation for similar reasons.

The relation distributions are computed from the concatenation of the three parts (train, dev and test) of each treebank from UD version 2.14 \cite{UD214}.